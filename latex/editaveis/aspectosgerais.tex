\chapter[Introdução]{Introdução}

O presente capítulo tem como objetivo apresentar as considerações iniciais sobre o Trabalho de Conclusão de Curso, que visa o desenvolvimento de uma aplicação web eficiente e centralizada para o gerenciamento de estágios na Universidade de Brasília (UnB). Inicialmente, será realizada uma Contextualização,
abordando o domínio de interesse deste trabalho, focado na gestão de estágios, e destacando a importância de um sistema centralizado que integre todas as etapas do processo de estágio, desde a inscrição até a conclusão.

Em seguida, será apresentada a Justificativa para a realização deste trabalho, esclarecendo as contribuições que a aplicação traz ao domínio da gestão acadêmica na UnB, especialmente no que tange à eficiência operacional,
acessibilidade e transparência dos dados relacionados aos estágios dos alunos.A proposta de desenvolvimento será orientada por princípios de desenvolvimento ágil, visando a construção de uma plataforma intuitiva e de fácil manutenção.

\section{Objetivos}
Este trabalho tem como objetivo desenvolver uma aplicação web eficiente 
e centralizada para o gerenciamento de estágios na UnB. A proposta visa automatizar e simplificar processos administrativos, 
proporcionando uma plataforma que permita gestão integrada de informações, monitoramento de atividades 
e interação facilitada entre os principais atores envolvidos: estudantes, professores orientadores, empresas concedentes 
e a instituição.

Os principais \textbf{objetivos específicos} desta aplicação incluem:
\begin{itemize}
    \item Criar um sistema que permita o cadastro, acompanhamento e gestão de estágios, com interface intuitiva para os usuários;
    \item Implementar funcionalidades para controle de documentos obrigatórios, como termos de compromisso e relatórios de estágio;
    \item Fornecer notificações automatizadas para prazos e entregas, melhorando a comunicação entre os envolvidos;
    \item Disponibilizar relatórios analíticos e indicadores de desempenho dos estágios para otimizar a tomada de decisões por parte da universidade;
    \item Assegurar um sistema robusto e seguro, que possa ser escalável e adaptado para outros departamentos ou universidades, caso necessário.
\end{itemize}
\section{Considerações sobre formatação básica do relatório}

A seguir são apresentadas as orientações básicas sobre a formatação do
documento. O modelo \LaTeX\ \textbf{já configura todas estas opções corretamente},
de modo que para os usuários deste modelo o texto de toda esta Seção é 
\textbf{meramente informativo}.

\subsection{Tipo de papel, fonte e margens}

Papel -- Na confecção do relatório deverá ser empregado papel branco no 
formato padrão A4 (21 cm x 29,7cm), com 75 a 90 g/m2.

Fonte -- Deve-se utilizar as fontes Arial ou Times New Roman no tamanho 12 
pra corpo do texto, com variações para tamanho 10 permitidas para a 
wpaginação, legendas e notas de rodapé. Em citações diretas de mais de três 
linhas utilizar a fonte tamanho 10, sem itálicos, negritos ou aspas. Os 
tipos itálicos são usados para nomes científicos e expressões estrangeiras, 
exceto expressões latinas.

Margens -- As margens delimitando a região na qual todo o texto deverá estar 
contido serão as seguintes: 

\begin{itemize}
	\item Esquerda: 03 cm;
	\item Direita	: 02 cm;
	\item Superior: 03 cm;
	\item Inferior: 02 cm. 
\end{itemize}

\subsection{Numeração de Páginas}

A contagem sequencial para a numeração de páginas começa a partir da 
primeira folha do trabalho que é a Folha de Rosto, contudo a numeração em 
si só deve ser iniciada a partir da primeira folha dos elementos textuais. 
Assim, as páginas dos elementos pré-textuais contam, mas não são numeradas 
e os números de página aparecem a partir da primeira folha dos elementos 
textuais, que se iniciam na Introdução. 

Os números devem estar em algarismos arábicos (fonte Times ou Arial 10) no 
canto superior direito da folha, a 02 cm da borda superior, sem traços, 
pontos ou parênteses. 

A paginação de Apêndices e Anexos deve ser contínua, dando seguimento ao 
texto principal.

\subsection{Espaços e alinhamento}

Para a monografia de TCC 01 e 02 o espaço entrelinhas do corpo do texto 
deve ser de 1,5 cm, exceto RESUMO, CITAÇÔES de mais de três linhas, NOTAS 
de rodapé, LEGENDAS e REFERÊNCIAS que devem possuir espaçamento simples. 
Ainda, ao se iniciar a primeira linha de cada novo parágrafo se deve 
tabular a distância de 1,25 cm da margem esquerda.

Quanto aos títulos das seções primárias da monografia, estes devem começar 
na parte superior da folha e separados do texto que o sucede, por um espaço 
de 1,5 cm entrelinhas, assim como os títulos das seções secundárias, 
terciárias. 

A formatação de alinhamento deve ser justificado, de modo que o texto fique 
alinhado uniformemente ao longo das margens esquerda e direita, exceto para 
CITAÇÕES de mais de três linhas que devem ser alinhadas a 04 cm da margem 
esquerda e REFERÊNCIAS que são alinhadas somente à margem esquerda do texto 
diferenciando cada referência.

\subsection{Quebra de Capítulos e Aproveitamento de Páginas}

Cada seção ou capítulo deverá começar numa nova pagina (recomenda-se que 
para texto muito longos o autor divida seu documento em mais de um arquivo 
eletrônico). 

Caso a última pagina de um capitulo tenha apenas um número reduzido de 
linhas (digamos 2 ou 3), verificar a possibilidade de modificar o texto 
(sem prejuízo do conteúdo e obedecendo as normas aqui colocadas) para 
evitar a ocorrência de uma página pouco aproveitada.

Ainda com respeito ao preenchimento das páginas, este deve ser otimizado, 
evitando-se espaços vazios desnecessários. 

Caso as dimensões de uma figura ou tabela impeçam que a mesma seja 
posicionada ao final de uma página, o deslocamento para a página seguinte 
não deve acarretar um vazio na pagina anterior. Para evitar tal ocorrência, 
deve-se reposicionar os blocos de texto para o preenchimento de vazios. 

Tabelas e figuras devem, sempre que possível, utilizar o espaço disponível 
da página evitando-se a \lq\lq quebra\rq\rq\ da figura ou tabela. 

\section{Cópias}

Nas versões do relatório para revisão da Banca Examinadora em TCC1 e TCC2, 
o aluno deve apresentar na Secretaria da FGA, uma cópia para cada membro da 
Banca Examinadora.

Após a aprovação em TCC2, o aluno deverá obrigatoriamente apresentar a 
versão final de seu trabalho à Secretaria da FGA na seguinte forma:

\begin{itemize}
	\item 01 cópia encadernada para arquivo na FGA;
	\item 01 cópia não encadernada (folhas avulsas) para arquivo na FGA;
	\item 01 cópia em CD de todos os arquivos empregados no trabalho.
\end{itemize}

A cópia em CD deve conter, além do texto, todos os arquivos dos quais se 
originaram os gráficos (excel, etc.) e figuras (jpg, bmp, gif, etc.) 
contidos no trabalho. Caso o trabalho tenha gerado códigos fontes e 
arquivos para aplicações especificas (programas em Fortran, C, Matlab, 
etc.) estes deverão também ser gravados em CD. 

O autor deverá certificar a não ocorrência de “vírus” no CD entregue a 
secretaria. 

