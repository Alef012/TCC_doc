\chapter{Desenvolvimento de Software}

\section*{Requisitos}

A elicitação de requisitos é uma etapa fundamental no processo de desenvolvimento de sistemas, sendo responsável por coletar, identificar e compreender as necessidades dos usuários e demais partes interessadas. Essa fase visa capturar as expectativas dos \textit{stakeholders} em relação ao sistema a ser desenvolvido, constituindo a base para a definição dos requisitos funcionais e não funcionais.

A elicitação de requisitos é um processo interativo, que pode envolver diversas técnicas para garantir uma compreensão completa e precisa do problema e das soluções esperadas. Entre as técnicas mais comuns estão entrevistas, questionários, \textit{workshops}, observação e \textit{brainstorming}. A escolha da técnica adequada depende do contexto do projeto, da disponibilidade dos participantes e do tipo de informação desejada.

Neste trabalho, optou-se por utilizar a técnica de \textit{brainstorming} como abordagem principal para a elicitação de requisitos. O \textit{brainstorming} é uma técnica criativa de geração de ideias que incentiva a participação ativa de todos os envolvidos, promovendo um ambiente livre de julgamentos e ideal para a coleta de sugestões espontâneas e diversas (Preece, Rogers e Sharp, 2013). Essa técnica mostrou-se adequada por permitir a rápida coleta de informações de diferentes pontos de vista, especialmente em um ambiente acadêmico com múltiplos atores, como alunos, professores e coordenadores.

A sessão de \textit{brainstorming} foi realizada com a participação de alunos que já passaram pelo processo de estágio e o professor que atua gerenciando esse processo. Durante a discussão, foram levantadas diversas funcionalidades esperadas para o sistema, sendo as principais listadas a seguir:


\begin{itemize}
    \item \textbf{Abertura de Solicitação de Estágio pelo Aluno:} O sistema deverá permitir que o aluno inicie o processo de estágio por meio da abertura de uma solicitação formal, preenchendo as informações necessárias diretamente na plataforma.
    
    \item \textbf{Indicação Automática de Professor Orientador:} Com base em critérios definidos previamente, como área de atuação ou disponibilidade, o sistema deverá sugerir automaticamente um professor orientador para acompanhar o estágio do aluno.
    
    \item \textbf{Gerenciamento da Assinatura do Professor Orientador:} O sistema deverá fornecer funcionalidades para que o professor orientador possa revisar, aprovar e assinar eletronicamente os documentos de estágio, garantindo praticidade e segurança.
    
    \item \textbf{Visualização de Alunos Vinculados a Cada Orientador:} Cada professor deverá ter acesso a uma lista com os alunos que o indicaram ou que foram automaticamente vinculados a ele como orientador, com dados relevantes do estágio em andamento.
    
    \item \textbf{Login Único:} O sistema deverá integrar-se com a plataforma institucional, utilizando autenticação por login único (\textit{Single Sign-On – SSO}), de modo que alunos e professores possam acessar o sistema com suas credenciais institucionais já existentes.
\end{itemize}

A partir dessa elicitação inicial, os requisitos serão refinados nas próximas etapas do desenvolvimento, como modelagem e validação, garantindo que o sistema atenda de forma eficaz às necessidades dos usuários.
