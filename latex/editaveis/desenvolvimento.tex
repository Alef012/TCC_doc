
\chapter{Desenvolvimento de Software}

\section*{Requisitos}

A elicitação de requisitos é uma etapa fundamental no processo de desenvolvimento de sistemas, sendo responsável por coletar, identificar e compreender as necessidades dos usuários e demais partes interessadas. Essa fase visa capturar as expectativas dos \textit{stakeholders} em relação ao sistema a ser desenvolvido, constituindo a base para a definição dos requisitos funcionais e não funcionais.

A elicitação de requisitos é um processo interativo, que pode envolver diversas técnicas para garantir uma compreensão completa e precisa do problema e das soluções esperadas. Entre as técnicas mais comuns estão entrevistas, questionários, \textit{workshops}, observação e \textit{brainstorming}. A escolha da técnica adequada depende do contexto do projeto, da disponibilidade dos participantes e do tipo de informação desejada.

Neste trabalho, optou-se por utilizar a técnica de \textit{brainstorming} como abordagem principal para a elicitação de requisitos. O \textit{brainstorming} é uma técnica criativa de geração de ideias que incentiva a participação ativa de todos os envolvidos, promovendo um ambiente livre de julgamentos e ideal para a coleta de sugestões espontâneas e diversas (Preece, Rogers e Sharp, 2013). Essa técnica mostrou-se adequada por permitir a rápida coleta de informações de diferentes pontos de vista, especialmente em um ambiente acadêmico com múltiplos atores, como alunos, professores e coordenadores.

A sessão de \textit{brainstorming} foi realizada com a participação de alunos que já passaram pelo processo de estágio e o professor que atua gerenciando esse processo. Durante a discussão, foram levantadas diversas funcionalidades esperadas para o sistema, sendo as principais listadas a seguir:


\begin{itemize}
    \item \textbf{Abertura de Solicitação de Estágio pelo Aluno:} O sistema deverá permitir que o aluno inicie o processo de estágio por meio da abertura de uma solicitação formal, preenchendo as informações necessárias diretamente na plataforma.
    
    \item \textbf{Indicação Automática de Professor Orientador:} Com base em critérios definidos previamente, como área de atuação ou disponibilidade, o sistema deverá sugerir automaticamente um professor orientador para acompanhar o estágio do aluno.
    
    \item \textbf{Gerenciamento da Assinatura do Professor Orientador:} O sistema deverá fornecer funcionalidades para que o professor orientador possa revisar, aprovar e assinar eletronicamente os documentos de estágio, garantindo praticidade e segurança.
    
    \item \textbf{Visualização de Alunos Vinculados a Cada Orientador:} Cada professor deverá ter acesso a uma lista com os alunos que o indicaram ou que foram automaticamente vinculados a ele como orientador, com dados relevantes do estágio em andamento.
    
    \item \textbf{Login Único:} O sistema deverá integrar-se com a plataforma institucional, utilizando autenticação por login único (\textit{Single Sign-On – SSO}), de modo que alunos e professores possam acessar o sistema com suas credenciais institucionais já existentes.
\end{itemize}

A partir dessa elicitação inicial, os requisitos serão refinados nas próximas etapas do desenvolvimento, como modelagem e validação, garantindo que o sistema atenda de forma eficaz às necessidades dos usuários.


\section{Priorização de Requisitos com a Técnica MoSCoW}

Após a etapa de elicitação, torna-se essencial organizar e classificar os requisitos identificados de acordo com sua importância e urgência. Essa priorização orienta o desenvolvimento incremental do sistema, garantindo que as funcionalidades mais críticas sejam implementadas primeiro, otimizando o uso de recursos e atendendo às expectativas principais dos \textit{stakeholders} \cite{pressman2016engenharia}.

Neste trabalho, foi adotada a técnica \textbf{MoSCoW} para a priorização dos requisitos. Essa técnica, bastante utilizada em metodologias ágeis, categoriza os requisitos em quatro grupos principais: \textit{Must have}, \textit{Should have}, \textit{Could have} e \textit{Won’t have for now} \cite{clegg1994moscow, wiegers2013software}.

\begin{itemize}
    \item \textbf{Must have (Deve ter):} Requisitos essenciais para o funcionamento mínimo do sistema. Sem esses itens, o sistema não atende ao seu propósito.
    \item \textbf{Should have (Deveria ter):} Requisitos importantes, mas não críticos. Sua ausência pode ser contornada temporariamente sem impactar significativamente o uso do sistema.
    \item \textbf{Could have (Poderia ter):} Requisitos desejáveis, que agregam valor, mas são menos prioritários. Podem ser implementados caso haja tempo e recursos disponíveis.
    \item \textbf{Won’t have for now (Não terá por enquanto):} Requisitos identificados, mas que não serão incluídos nesta versão do sistema, podendo ser considerados em futuras iterações.
\end{itemize}

A Tabela~\ref{tab:priorizacao} apresenta a priorização dos requisitos levantados na etapa de elicitação, com base na técnica MoSCoW.

\begin{table}[H]
\centering
\caption{Priorização dos requisitos segundo a técnica MoSCoW}
\label{tab:priorizacao}
\begin{tabular}{|p{6cm}|c|p{6cm}|}
\hline
\textbf{Requisito} & \textbf{Prioridade} & \textbf{Justificativa} \\
\hline
Abertura de Solicitação de Estágio pelo Aluno & Must have & Essencial para iniciar o processo de estágio; funcionalidade central do sistema. \\
\hline
Indicação Automática de Professor Orientador & Should have & Importante para agilizar o processo, mas pode ser feita manualmente em um primeiro momento. \\
\hline
Gerenciamento da Assinatura do Professor Orientador & Must have & Fundamental para garantir a formalização do processo de estágio e reduzir burocracia. \\
\hline
Visualização de Alunos Vinculados a Cada Orientador & Should have & Facilita o acompanhamento, mas o processo pode iniciar sem essa funcionalidade completa. \\
\hline
Login Único (SSO) & Could have & Facilita o acesso, mas pode ser implementado após uma versão inicial com login próprio. \\
\hline
\end{tabular}
\end{table}

Essa priorização permitirá que a equipe de desenvolvimento foque inicialmente nos requisitos que compõem o núcleo funcional do sistema, entregando uma primeira versão utilizável o quanto antes. Os requisitos classificados como \textit{Should have} e \textit{Could have} poderão ser incorporados gradualmente, conforme a evolução do projeto e a disponibilidade de recursos \cite{somerville2011engenharia}.


\section*{Tecnologias}
Esse sessão está separada para a descrição das tecnologias utilizadas no desenvolvimento


\section*{Arquitetura}
Aqui a descrição da arquitetura do projeto

